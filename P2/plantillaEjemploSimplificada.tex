\input{preambuloSimple.tex}

%----------------------------------------------------------------------------------------
%	TÍTULO Y DATOS DEL ALUMNO
%----------------------------------------------------------------------------------------

\title{
\normalfont \normalsize
\texttt{{\bf Visión Por Computador (2015-2016)} \\ Grado en Ingeniería Informática \\ Universidad de Granada} \\ [25pt] % Your university, school and/or department name(s)
\horrule{0.5pt} \\[0.4cm] % Thin top horizontal rule
\huge{ Cuestionario 2} \\ % The assignment title
\horrule{2pt} \\[0.5cm] % Thick bottom horizontal rule
}

\author{Mª Cristina Heredia Gómez} % Nombre y apellidos
%latexmk -shell-escape -pdf -pvc plantillaEjemploSimplificada.tex ; latexmk -C
\date{\normalsize\today} % Incluye la fecha actual

%----------------------------------------------------------------------------------------
% DOCUMENTO
%----------------------------------------------------------------------------------------

\begin{document}

\maketitle % Muestra el Título

\newpage %inserta un salto de página

\tableofcontents % para generar el índice de contenidos

\listoffigures

%\listoftables

\newpage

%NOTA: en caso de problema al compilar, compruebe que tiene el paquete: texlive-babel-spanish.noarch  \\


\newpage

%----------------------------------------------------------------------------------------
%	Cuestión 1
%----------------------------------------------------------------------------------------
\section{¿Identificar la\/s diferencia\/s esencial\/es entre el plano afín y el plano
proyectivo? ¿Cuáles son sus consecuencias? Justificar la contestación.}
La principal diferencia entre un plano afín y un plano proyectivo, es que en un plano afín tenemos que el paralelismo es una
relación de equivalencia entre todas las líneas de tal plano, mientras que en un plano proyectivo, tenemos que cualesquiera dos líneas
se cortan en un punto, único. Éste punto puede ser un $"$punto en el infinito$"$ , siendo éste donde se cortan las líneas con una línea paralela.
\newline
Por esta razón, en un plano proyectivo tendremos que dadas dos rectas diferentes cualesquiera, estas tienen esactamente un punto en común,
mientras que en un plano afín, dadas dos rectas diferentes cualesquiera, éstas no tendrán ningún punto en común.

\begin{figure}[H] %con el [H] le obligamos a situar aquí la figura
\centering
\includegraphics[scale=0.4]{transformations.png}  %el parámetro scale permite agrandar o achicar la imagen. En el nombre de archivo puede especificar directorios
\label{figura1}
\caption{figura 3.45 del libro Richard Szeliski }
\end{figure}
Podemos pasar de un plano proyectivo a uno afín, quitando líneas no paralelas y los puntos que contienen, al igual que podemos pasar
de un plano afín a un plano proyectivo, introduciendo una recta en el infinito. \newline
Como podemos ver, la principal consecuencia que tenemos es que ,en un plano afín mantenemos el paralelismo, mientras que en un plano proyectivo
mantenemos las líneas rectas y las distancias pero \textbf{el paralelismo se pierde}.
%----------------------------------------------------------------------------------------
%	Cuestión 2
%----------------------------------------------------------------------------------------
\section{Verificar que en coordenadas homogéneas el vector de la recta definida por
dos puntos puede calcularse como el producto vectorial de los vectores de los
puntos ($l = x\cdot {x}'$). De igual modo el punto intersección de dos rectas l y
 ${l}'$ está dado por $x = l\cdot {l}'$}

Sea $\varphi$ la aplicación que lleva de un plano afín a un plano proyectivo:
 \[\varphi : A^{2}\rightarrow P^{2} / (x,y)\rightarrow (x,y,1)\]
 Sean $P_{1}=(x_{1},y_{1})$ y $P_{2}=(x_{2},y_{2})$ $\epsilon$ A, dos puntos cualesquiera. Entonces sabemos que un vector de la recta definida
 por esos dos puntos, es:
 \[\vec{v}_{r}=\overrightarrow{P_{2}P_{1}}\]
donde  $  \vec{v}_{r^{_{1}}}=x_{1}-x_{2} $ y $ \vec{v}_{r^{_{2}}}=y_{1}-y_{2}$
entonces, sustituyendo en la ecuación de la forma contigua, tenemos:
\[\frac{x-x_{1}}{\vec{v}_{r^{_{1}}}}=\frac{y-y_{1}}{\vec{v}_{r^{_{2}}}}\Rightarrow \frac{x-x_{1}}{x_{1}-x_{2}}=\frac{y-y_{1}}{y_{1}-y_{2}}\]
y despejando:
\[(x-x_{1})(y_{1}-y_{2})=(x_{1}-x_{2})(y-y_{1})\]
\[x\cdot y_{1}-x\cdot y_{2}-x_{1}\cdot y_{1}+ x_{1}\cdot y_{2}=x_{1}\cdot y-x_{1}\cdot y_{1}-y\cdot x_{2}+y_{1}\cdot x_{2} \Rightarrow \]
\[x(y_{1}-y_{2})+x_{1}y_{2}=y(x_{1}-x_{2})+y_{1}x_{2} \Rightarrow \]
\[x(y_{1}-y_{2})-y(x_{1}-x_{2})+(x_{1}y_{2}-y_{1}x_{2})=0\]
y por tanto el vector de la recta será: \[ \vec{v}_{r}=(y_{1}-y_{2}, -x_{1}+x_{2}, x_{1}y_{2}-y_{1}x_{2})\]
\newline
Calculemos ahora el producto vectorial de los dos puntos:
\[
\vec{P1}\vec{P2}=\begin{vmatrix}
\vec{i} &\vec{j}  &\vec{k} \\
 x_{1}& y_{1} &1 \\
 x_{2}& y_{2} &1
\end{vmatrix}=\begin{vmatrix}
 y_{1}&1 \\
 y_{2}&1
\end{vmatrix}\vec{i}-\begin{vmatrix}
x_{1} & 1\\
 x_{2}&1
\end{vmatrix}\vec{j}+\begin{vmatrix}
 x_{1}&y_{1} \\
 x_{2}&y_{2}
\end{vmatrix}\vec{k}=
\]
\[=(y_{1}-y_{2})\vec{i}-(x_{1}-x_{2})\vec{j}+(x_{1}y_{2}-x_{2}y_{1})\vec{k}\]
por lo que tenemos que las coordenadas del vector de la recta serían:
\[ \vec{v}_{r}=(y_{1}-y_{2}, -x_{1}+x_{2}, x_{1}y_{2}-y_{1}x_{2})\]

obteniendo lo mismo que en el caso anterior. Por lo tanto, tendríamos que el vector de la recta definida por
dos puntos puede calcularse como el producto vectorial de tales puntos.
%----------------------------------------------------------------------------------------
%	Cuestión 3
%----------------------------------------------------------------------------------------
\section{Sean x y l un punto y una recta respectivamente en un plano proyectivo P1
y suponemos que la recta l pasa por el punto x, es decir ${l}'^{T}x=0$. Sean ${x}'$ y ${l}'$
un punto y una recta del plano proyectivo ${P}'$ donde al igual que antes ${l}'^{T}{x}'=0$.
Supongamos que existe un homografía de puntos H entre ambos planos proyectivos,
es decir ${x}'=Hx$. Deducir de las ecuaciones anteriores la expresión para la
homografía G que relaciona los vectores de las rectas, es decir G tal que ${l}'=Gl$.
Justificar la respuesta}

%----------------------------------------------------------------------------------------
%	Cuestión 4
%----------------------------------------------------------------------------------------
\section{Suponga la imagen de un plano en donde el vector l=(l 1 ,l 2 ,l 3 ) representa
la proyección de la recta del infinito del plano en la imagen. Sabemos que si
conseguimos aplicar a nuestra imagen una homografía G tal que si ${l}'=Gl$,siendo
${l}'T=(0,0,1)$ entonces habremos rectificado nuestra imagen llevándola de nuevo
al plano afín. Suponiendo que la recta definida por l no pasa por el punto (0,0)
del plano imagen. Encontrar la homografía G. Justificar la respuesta}

%----------------------------------------------------------------------------------------
%	Cuestión 5
%----------------------------------------------------------------------------------------
\section{Identificar los movimientos elementales (traslación, giro, escala,
cizalla, proyectivo) representados por las homografías H1, H2, H3 y H4:}

\begin{equation}
 H1 =
 \begin{pmatrix}
   1 & 0 & 3\\ 0 & 1 & 5\\ 0 & 0 & 1
 \end{pmatrix}
 \begin{pmatrix}
   0.5 & 0 & 0\\ 0 & 0.3 & 0\\ 0 & 0 & 1
 \end{pmatrix}
 \begin{pmatrix}
   1 & 3 & 0\\ 0 & 1 & 0\\ 0 & 0 & 1
 \end{pmatrix}
 \end{equation}
 \begin{equation}
 H2  =
 \begin{pmatrix}
   0 & 1 & -3\\ -1 & 0 & 2\\ 0 & 0 & 1
 \end{pmatrix}
 \begin{pmatrix}
   2 & 0 & 0\\ 2 & 2 & 0\\ 0 & 0 & 1
 \end{pmatrix}
\end{equation}
\begin{equation}
 H3  =
 \begin{pmatrix}
   1 & 0.5 & 0\\ 0.5 & 2 & 0\\ 0 & 0 & 1
 \end{pmatrix}
 \begin{pmatrix}
   1 & 0 & 0\\ 0 & 1 & 0\\ -1 & 0 & 1
 \end{pmatrix}
 \end{equation}
 \begin{equation}
 H4  =
 \begin{pmatrix}
   2 & 0 & 3\\ 0 & 2 & -1\\ 0 & 1 & 2
 \end{pmatrix}
\end{equation}

%----------------------------------------------------------------------------------------
%	Cuestión 6
%----------------------------------------------------------------------------------------
\section{¿Cuáles son las propiedades necesarias y suficientes para que una matriz
defina una homografía entre planos? Justificar la respuesta}

%----------------------------------------------------------------------------------------
%	Cuestión 7
%----------------------------------------------------------------------------------------
\section{¿Qué propiedades de la geometría de un plano quedan invariantes si se aplicauna homografía general sobre él?
 Justificar la respuesta.}

%----------------------------------------------------------------------------------------
%	Cuestión 8
%----------------------------------------------------------------------------------------
\section{¿Cuál es la deformación geométrica más fuerte que se puede producir sobre
la imagen de un plano por el punto de vista de la cámara? Justificar la respuesta.}


%----------------------------------------------------------------------------------------
%	Cuestión 9
%----------------------------------------------------------------------------------------
\section{¿Qué información de la imagen usa el detector de Harris para seleccionar
puntos? ¿El detector de Harris detecta patrones geométricos o fotométricos?
Justificar la contestación.}


%----------------------------------------------------------------------------------------
%	Cuestión 10
%----------------------------------------------------------------------------------------
\section{¿Sería adecuado usar como descriptor de un punto Harris los valores de
los píxeles de su región de soporte? En caso positivo identificar cuando y
justificar la respuesta}


%----------------------------------------------------------------------------------------
%	Cuestión 11
%----------------------------------------------------------------------------------------
\section{¿Qué información de la imagen se codifica en el descriptor de SIFT?
Justificar la contestación.}

%----------------------------------------------------------------------------------------
%	Cuestión 12
%----------------------------------------------------------------------------------------
\section{Describa un par de criterios que sirvan para establecer correspondencias
(matching) entre descriptores de regiones extraídos de dos imágenes. Justificar
la idoneidad de los mismos}

%----------------------------------------------------------------------------------------
%	Cuestión 13
%----------------------------------------------------------------------------------------
\section{Cual es el objetivo principal en el uso de la técnica RANSAC. Justificar
la respuesta}

%----------------------------------------------------------------------------------------
%	Cuestión 14
%----------------------------------------------------------------------------------------
\section{¿Si tengo 4 imágenes de una escena de manera que se solapan la 1-2, 2-3
y 3-4. ¿Cuál es el número mínimo de puntos en correspondencias necesarios para
montar un mosaico? Justificar la respuesta}

%----------------------------------------------------------------------------------------
%	Cuestión 15
%----------------------------------------------------------------------------------------
\section{En la confección de un mosaico con proyección rectangular es esperable
que aparezcan deformaciones de la realidad. ¿Cuáles y porqué?.¿Bajo qué
condiciones esas deformaciones podrían desaparecer? Justificar la respuesta}

\end{document}
