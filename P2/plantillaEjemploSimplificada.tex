\input{preambuloSimple.tex}

%----------------------------------------------------------------------------------------
%	TÍTULO Y DATOS DEL ALUMNO
%----------------------------------------------------------------------------------------

\title{
\normalfont \normalsize
\texttt{{\bf Visión Por Computador (2015-2016)} \\ Grado en Ingeniería Informática \\ Universidad de Granada} \\ [25pt] % Your university, school and/or department name(s)
\horrule{0.5pt} \\[0.4cm] % Thin top horizontal rule
\huge{ Cuestionario 2} \\ % The assignment title
\horrule{2pt} \\[0.5cm] % Thick bottom horizontal rule
}

\author{Mª Cristina Heredia Gómez} % Nombre y apellidos
%latexmk -shell-escape -pdf -pvc plantillaEjemploSimplificada.tex ; latexmk -C
\date{\normalsize\today} % Incluye la fecha actual

%----------------------------------------------------------------------------------------
% DOCUMENTO
%----------------------------------------------------------------------------------------

\begin{document}

\maketitle % Muestra el Título

\newpage %inserta un salto de página

\tableofcontents % para generar el índice de contenidos

\listoffigures

%\listoftables

\newpage

%NOTA: en caso de problema al compilar, compruebe que tiene el paquete: texlive-babel-spanish.noarch  \\


\newpage

%----------------------------------------------------------------------------------------
%	Cuestión 1
%----------------------------------------------------------------------------------------
\section{¿Identificar la\/s diferencia\/s esencial\/es entre el plano afín y el plano
proyectivo? ¿Cuáles son sus consecuencias? Justificar la contestación.}
La principal diferencia entre un plano afín y un plano proyectivo, es que en un plano afín tenemos que el paralelismo es una
relación de equivalencia entre todas las líneas de tal plano, mientras que en un plano proyectivo, tenemos que cualesquiera dos líneas
se cortan en un punto, único. Éste punto puede ser un $"$punto en el infinito$"$ , siendo éste donde se cortan las líneas con una línea paralela.
\newline
Por esta razón, en un plano proyectivo tendremos que dadas dos rectas diferentes cualesquiera, estas tienen esactamente un punto en común,
mientras que en un plano afín, dadas dos rectas diferentes cualesquiera, éstas no tendrán ningún punto en común.

\begin{figure}[H] %con el [H] le obligamos a situar aquí la figura
\centering
\includegraphics[scale=0.4]{transformations.png}  %el parámetro scale permite agrandar o achicar la imagen. En el nombre de archivo puede especificar directorios
\label{figura1}
\caption{figura 3.45 del libro Richard Szeliski }
\end{figure}
Podemos pasar de un plano proyectivo a uno afín, quitando líneas no paralelas y los puntos que contienen, al igual que podemos pasar
de un plano afín a un plano proyectivo, introduciendo una recta en el infinito. \newline
Como podemos ver, la principal consecuencia que tenemos es que ,en un plano afín mantenemos el paralelismo, mientras que en un plano proyectivo
mantenemos las líneas rectas y las distancias pero \textbf{el paralelismo se pierde}.
%----------------------------------------------------------------------------------------
%	Cuestión 2
%----------------------------------------------------------------------------------------
\section{Verificar que en coordenadas homogéneas el vector de la recta definida por
dos puntos puede calcularse como el producto vectorial de los vectores de los
puntos ($l = x\cdot {x}'$). De igual modo el punto intersección de dos rectas l y
 ${l}'$ está dado por $x = l\cdot {l}'$}

Sea $\varphi$ la aplicación que lleva de un plano afín a un plano proyectivo:
 \[\varphi : A^{2}\rightarrow P^{2} / (x,y)\rightarrow (x,y,1)\]
 Sean $P_{1}=(x_{1},y_{1})$ y $P_{2}=(x_{2},y_{2})$ $\epsilon$ A, dos puntos cualesquiera. Entonces sabemos que un vector de la recta definida
 por esos dos puntos, es:
 \[\vec{v}_{r}=\overrightarrow{P_{2}P_{1}}\]
donde  $  \vec{v}_{r^{_{1}}}=x_{1}-x_{2} $ y $ \vec{v}_{r^{_{2}}}=y_{1}-y_{2}$
entonces, sustituyendo en la ecuación de la forma contigua, tenemos:
\[\frac{x-x_{1}}{\vec{v}_{r^{_{1}}}}=\frac{y-y_{1}}{\vec{v}_{r^{_{2}}}}\Rightarrow \frac{x-x_{1}}{x_{1}-x_{2}}=\frac{y-y_{1}}{y_{1}-y_{2}}\]
y despejando:
\[(x-x_{1})(y_{1}-y_{2})=(x_{1}-x_{2})(y-y_{1})\]
\[x\cdot y_{1}-x\cdot y_{2}-x_{1}\cdot y_{1}+ x_{1}\cdot y_{2}=x_{1}\cdot y-x_{1}\cdot y_{1}-y\cdot x_{2}+y_{1}\cdot x_{2} \Rightarrow \]
\[x(y_{1}-y_{2})+x_{1}y_{2}=y(x_{1}-x_{2})+y_{1}x_{2} \Rightarrow \]
\[x(y_{1}-y_{2})-y(x_{1}-x_{2})+(x_{1}y_{2}-y_{1}x_{2})=0\]
y por tanto el vector de la recta será: \[ \vec{v}_{r}=(y_{1}-y_{2}, -x_{1}+x_{2}, x_{1}y_{2}-y_{1}x_{2})\]
\newline
Calculemos ahora el producto vectorial de los dos puntos:
\[
\vec{P1}\vec{P2}=\begin{vmatrix}
\vec{i} &\vec{j}  &\vec{k} \\
 x_{1}& y_{1} &1 \\
 x_{2}& y_{2} &1
\end{vmatrix}=\begin{vmatrix}
 y_{1}&1 \\
 y_{2}&1
\end{vmatrix}\vec{i}-\begin{vmatrix}
x_{1} & 1\\
 x_{2}&1
\end{vmatrix}\vec{j}+\begin{vmatrix}
 x_{1}&y_{1} \\
 x_{2}&y_{2}
\end{vmatrix}\vec{k}=
\]
\[=(y_{1}-y_{2})\vec{i}-(x_{1}-x_{2})\vec{j}+(x_{1}y_{2}-x_{2}y_{1})\vec{k}\]
por lo que tenemos que las coordenadas del vector de la recta serían:
\[ \vec{v}_{r}=(y_{1}-y_{2}, -x_{1}+x_{2}, x_{1}y_{2}-y_{1}x_{2})\]

obteniendo lo mismo que en el caso anterior. Por lo tanto, tendríamos que el vector de la recta definida por
dos puntos puede calcularse como el producto vectorial de tales puntos.
%----------------------------------------------------------------------------------------
%	Cuestión 3
%----------------------------------------------------------------------------------------
\section{Sean x y l un punto y una recta respectivamente en un plano proyectivo P1
y suponemos que la recta l pasa por el punto x, es decir $l^{T}x=0$. Sean ${x}'$ y ${l}'$
un punto y una recta del plano proyectivo ${P}'$ donde al igual que antes ${l}'^{T}{x}'=0$.
Supongamos que existe un homografía de puntos H entre ambos planos proyectivos,
es decir ${x}'=Hx$. Deducir de las ecuaciones anteriores la expresión para la
homografía G que relaciona los vectores de las rectas, es decir G tal que $ {l}'=Gl$.\newline
Justificar la respuesta}
Sean  $l^{T}\cdot x=0$ tal que x,l $\epsilon$ P1 y sean ${l}'^{T}\cdot {x}'=0$ tal que ${x}'$, ${l}'$ $\epsilon$ ${P}'$
Si \[\exists H / {x}'=Hx\] $\Rightarrow $ ¿G tal que  ${l}'=Gl$?
Tenemos que: \newline
\[{l}'{x}'={l}'^{T}Hx=(H^{T}{l}')^{T}x=lx=0\]
$\Rightarrow $  como $(H^{T}{l}')^{T}x=lx$
entonces tendremos que :  ${l}'=H^{-T}l$ y por tanto, $H^{-T}$ será la homografía G que buscábamos y que relaciona
los vectores de las rectas.
%----------------------------------------------------------------------------------------
%	Cuestión 4
%----------------------------------------------------------------------------------------
\section{Suponga la imagen de un plano en donde el vector l=(l 1 ,l 2 ,l 3 ) representa
la proyección de la recta del infinito del plano en la imagen. Sabemos que si
conseguimos aplicar a nuestra imagen una homografía G tal que si ${l}'=Gl$,siendo
${l}'T=(0,0,1)$ entonces habremos rectificado nuestra imagen llevándola de nuevo
al plano afín. Suponiendo que la recta definida por l no pasa por el punto (0,0)
del plano imagen. Encontrar la homografía G. Justificar la respuesta}

%----------------------------------------------------------------------------------------
%	Cuestión 5
%----------------------------------------------------------------------------------------
\section{Identificar los movimientos elementales (traslación, giro, escala,
cizalla, proyectivo) representados por las homografías H1, H2, H3 y H4:}

\begin{equation}
 H1 =
 \begin{pmatrix}
   1 & 0 & 3\\ 0 & 1 & 5\\ 0 & 0 & 1
 \end{pmatrix}
 \begin{pmatrix}
   0.5 & 0 & 0\\ 0 & 0.3 & 0\\ 0 & 0 & 1
 \end{pmatrix}
 \begin{pmatrix}
   1 & 3 & 0\\ 0 & 1 & 0\\ 0 & 0 & 1
 \end{pmatrix}
 \end{equation}
 \begin{equation}
 H2  =
 \begin{pmatrix}
   0 & 1 & -3\\ -1 & 0 & 2\\ 0 & 0 & 1
 \end{pmatrix}
 \begin{pmatrix}
   2 & 0 & 0\\ 2 & 2 & 0\\ 0 & 0 & 1
 \end{pmatrix}
\end{equation}
\begin{equation}
 H3  =
 \begin{pmatrix}
   1 & 0.5 & 0\\ 0.5 & 2 & 0\\ 0 & 0 & 1
 \end{pmatrix}
 \begin{pmatrix}
   1 & 0 & 0\\ 0 & 1 & 0\\ -1 & 0 & 1
 \end{pmatrix}
 \end{equation}
 \begin{equation}
 H4  =
 \begin{pmatrix}
   2 & 0 & 3\\ 0 & 2 & -1\\ 0 & 1 & 2
 \end{pmatrix}
\end{equation}

\subsubsection{H1}
tenemos una translación, un escalado y un movimiento afín.
La primera matriz
 \begin{equation}
\begin{pmatrix}
  1 & 0 & 3\\ 0 & 1 & 5\\ 0 & 0 & 1
\end{pmatrix}
 \end{equation}
representa una translación, con $t_{x}=3$ y $t_{y}=5$ , ya que tiene la forma:
\begin{equation}
\begin{pmatrix}
 1 & 0 & t_{x}\\ 0 & 1 & t_{y}\\ 0 & 0 & 1
\end{pmatrix}
\end{equation}
La segunda matriz, representa un escalado, con $S_{x}=0.5\neq 0$ y $S_{y}=0.3\neq 0$ , ya que es  de la forma:
\begin{equation}
\begin{pmatrix}
 S_{x} & 0 & 0\\ 0 & S_{y} & 0\\ 0 & 0 & 1
\end{pmatrix}
\end{equation}

Por último, la tercera matriz representa un movimiento afín, con  $C_{x}=3$ y $C_{y}=0$, mediante el cual añade deformación(pierde ángulos), aunque mantiene las rectas.
Creo que es un movimiento afín por ser de la forma:
\begin{equation}
\begin{pmatrix}
 1 & C_{x} & 0\\ C_{y} & 1 & 0\\ 0 & 0 & 1
\end{pmatrix}
\end{equation} descartamos la posibilidad de que pudiera ser una rotación, dado que aunque $\begin{vmatrix} 1 &1 \\ 0&3 \end{vmatrix}=1$, tenemos que
 $R\cdot R^{T}\neq id$

\subsubsection{H2}
La primera matriz:
\begin{equation}
\begin{pmatrix}
  0 & 1 & -3\\ -1 & 0 & 2\\ 0 & 0 & 1
\end{pmatrix}
\end{equation} representa un giro+translación, con $\cos \theta =0$ y $\sin \theta =1$ ya que es de la forma:
\begin{equation}
\begin{pmatrix}
  \cos \theta & \sin \theta &t_{x} \\ -\sin \theta & \cos \theta &t_{y} \\ 0 & 0 & 1
\end{pmatrix}
\end{equation}
y cumple que  $\begin{vmatrix} 0 &1 \\ -1&0 \end{vmatrix}=1$ y que $\bigl(\begin{smallmatrix} 0 &1 \\ -1 &0 \end{smallmatrix}\bigr)\cdot \bigl(\begin{smallmatrix} 0 &1 \\ -1 &0 \end{smallmatrix}\bigr)^{T}=\bigl(\begin{smallmatrix} 1 &0 \\ 0 &1 \end{smallmatrix}\bigr)=id$.
\newline
La segunda matriz:
\begin{equation}
\begin{pmatrix}
  2 & 0 & 0\\ 2 & 2 & 0\\ 0 & 0 & 1
\end{pmatrix}
\end{equation} representa una transformación afín con $C_{x}=0$ y $C_{y}=2$.

\subsubsection{H3}
La primera matriz:
\begin{equation}
\begin{pmatrix}
  1 & 0.5 & 0\\ 0.5 & 2 & 0\\ 0 & 0 & 1
\end{pmatrix}
\end{equation} representa un movimiento afín, con $C_{x}=0.5$ y $C_{y}=0.5$, mientras que la segunda matriz:
\begin{equation}
\begin{pmatrix}
  1 & 0 & 0\\ 0 & 1 & 0\\ -1 & 0 & 1
\end{pmatrix}
\end{equation} representa una transformación proyectiva, ya que modifica los puntos $"$ introduciendo perspectiva $"$ .

\subsubsection{H4}
Por último, la matriz:
\begin{equation}
H4  =
\begin{pmatrix}
  2 & 0 & 3\\ 0 & 2 & -1\\ 0 & 1 & 2
\end{pmatrix}
\end{equation}

que es de la forma:
\[\begin{pmatrix} a_{11}&a_{12} &a_{13} \\ a_{21}&a_{22} &a_{23} \\ a_{31}&a_{32} &a_{33} \end{pmatrix}\]
(observamos que su última fila es $\neq$ (0,0,1) )por tanto, aplica una transformación proyectiva sobre los puntos de la imágen.

%----------------------------------------------------------------------------------------
%	Cuestión 6
%----------------------------------------------------------------------------------------
\section{¿Cuáles son las propiedades necesarias y suficientes para que una matriz
defina una homografía entre planos? Justificar la respuesta}
Una homografía H, es:  $ {u}'=H\cdot u $
\newline    tal que si multiplicamos un vector de puntos (u) por H,
obtenemos un vector nuevo de puntos ({u}'), calculados a partir del anterior, aplicándole una serie
de transformaciones. \newline
Hablemos de sus propiedades más imporantes. Sabemos que todo movimiento puede ser
representado con varios puntos, sin embargo, si vamos a calcular un homografía, necesitaremos que esos puntos
estén en correspondencias en las dos imágenes, o la homografía no será correcta. \newline \newline
Otro detalle es que, a pesar de que los puntos pueden ser cualesquiera, no deberá de haber más de dos alineados si queremos
obtener un resultado (más o menos) bueno. \newline
Además, una homografía debe tener tamaño 3x3 y debe de tener det $\neq$ 0 , ya que debe tener inversa, pues igual que es posible pasar de
una imágen1 a una imágen2 a través de H, tiene que ser posible pasar de una imágen2 a la imágen1, a través de $H^{-1}$.


%----------------------------------------------------------------------------------------
%	Cuestión 7
%----------------------------------------------------------------------------------------
\section{¿Qué propiedades de la geometría de un plano quedan invariantes si se aplica una homografía general sobre él?
 Justificar la respuesta.}
Una homografía general pueden ser varias transformaciones, no es una transformación geométrica concreta.
Por lo tanto, hay propiedades como la proyección o el paralelismo
que según la homografía, podrían o no conservarse. \newline
Ésto nos lleva a pensar ¿entonces qué se conserva? \newline
Resulta evidente que aplicándole una transformación a una imágen, lo que estaba alineado en la imágen original lo sigue estando
en la transformada. Por tanto , ahí tenemos algo que se conserva: \textbf{las rectas}.\newline Si tenemos que unas rectas se intersecaban
en la imágen original, entonces también lo harán en la transformada. \newline
Algo parecido ocurre con \textbf{las distancias}, que también se conservan, de tal forma que lo que estaba a cierta distancia en la imágen original
sigue estando a esa cierta distancia en la transformada.

%----------------------------------------------------------------------------------------
%	Cuestión 8
%----------------------------------------------------------------------------------------
\section{¿Cuál es la deformación geométrica más fuerte que se puede producir sobre
la imagen de un plano por el punto de vista de la cámara? Justificar la respuesta.}
Yo creo la deformación geométrica más fuerte que se puede producir sobre una imágen en un plano
por el punto de vista de la cámara está provocado por el zoom, ya que a más zoom tendré
menos arco de visión, esto es, menos arcos de luz, suponiendo que hablamos de una cámara pinhole.\newline
Por tanto, mientras que si la cámara está lejos proyecta claramente un plano de la imágen,
cuando está muy cerca lo que proyecta es un plano 3D, ya que si vamos haciendo zoom nos vamos centrando
en un plano más paralelo al plano imágen, mientras que si nos alejamos vamos perdiendo ese plano paralelo.
%----------------------------------------------------------------------------------------
%	Cuestión 9
%----------------------------------------------------------------------------------------
\section{¿Qué información de la imagen usa el detector de Harris para seleccionar
puntos? ¿El detector de Harris detecta patrones geométricos o fotométricos?
Justificar la contestación.}
El detector de Harris selecciona puntos basándose en el gradiente.

\[H=\begin{pmatrix} \sum_{w}(\frac{\partial I}{\partial x})^{2} & \sum_{w}(\frac{\partial I}{\partial x})(\frac{\partial I}{\partial y}) \\ \sum_{w}(\frac{\partial I}{\partial x})(\frac{\partial I}{\partial y})& \sum_{w}(\frac{\partial I}{\partial y})^{2} \end{pmatrix}\]

donde las derivadas son invariantes a giros.
Se fija en las regiones donde
hay cambios fotométricos (más variación luminosa) y cambios en la dirección. Para ello
usa dos valores ortogonales que nos indican cuando la variación del gradiente es máxima, permitiendo así detectar las esquinas.
\newline
El detector de Harris es monoescala, aunque se pueden hacer múltiples escalas y combinarlas poseriormente. \newline Además
el detector de Harris detecta \textbf{patrones fotométricos}, pues como hemos dicho antes, se basa en la magnitud del gradiente, y sin
embargo no detecta patrones geométricos, por lo que no nos da la información que podría obtener de alrededor de los puntos(no hace extracción de información)
, a diferencia
de otros detectores.


%----------------------------------------------------------------------------------------
%	Cuestión 10
%----------------------------------------------------------------------------------------
\section{¿Sería adecuado usar como descriptor de un punto Harris los valores de
los píxeles de su región de soporte? En caso positivo identificar cuando y
justificar la respuesta}
Yo creo que sí, ya que los valores de los píxeles al fin y al cabo son coordenadas de puntos
que me van a indicar si en una región dada de la imágen los puntos son relevantes o no lo son.\newline
Para ello, habrá que efectuar la suma sobre toda la región (donde tendremos hacia dónde apunta la dirección
más importante), eso sí, teniendo en cuenta que no haya variaciones en la dirección. \newline
Como la suma se hace en toda la ventana, será necesario ,aparte de dar las coordenadas de los puntos,
también dar la escala.

%----------------------------------------------------------------------------------------
%	Cuestión 11
%----------------------------------------------------------------------------------------
\section{¿Qué información de la imagen se codifica en el descriptor de SIFT?
Justificar la contestación.}
\textbf{Sift} es una técnica de detección, pero también de extraccción de información, a
diferencia de los puntos Harris. Se trata de un descriptor también basado en gradientes, que no se basa ni en
niveles de gris ni en intensidad luminosa. \newline
Lo que Sift hace es : teniendo los vectores directores de los gradientes para cada píxel, toma las direcciones
en las que el gradiente es mayor y con eso, construye un hibstograma. Para construir el hibstograma, vota a los números vecinos
, siguiendo algún criterio, como el vecino más cercano, por ejemplo.\newline
Teniendo en cuenta que hay 16 hibstogramas en total y que cada una tiene 8 números,  entonces tendremos que el descriptor Sift viene dado por 128
números, que son la información que se extrae para describir la región.

%----------------------------------------------------------------------------------------
%	Cuestión 12
%----------------------------------------------------------------------------------------
\section{Describa un par de criterios que sirvan para establecer correspondencias
(matching) entre descriptores de regiones extraídos de dos imágenes. Justificar
la idoneidad de los mismos}
En prácticas hemos usado dos: \textbf{fuerza bruta+cruce}, que lo que hace es comparar todos con todos; comparando todos los descriptores
de una imágen con el descriptor de un punto de la otra (para cada punto), y quedarnos con los mejores. La opción de cruce además lo que añade
es que se compare izquierda con derecha y derecha con izquierda.
\newline
Otro matcher que hemos usado en pŕacticas ha sido \textbf{Flann}, que son las siglas de librería rápida para aproximar vecinos cercanos.\newline
Ésta librería contiene varios algoritmos optimizados, que se pueden seleccionar a la hora de hacer el matching. Pero lo importante,
es que aproxima vecinos cercanos, es decir, que en lugar de comparar todos con todos como el anterior, utiliza los descriptores que
están más cerca. Para establecer qué es estar cerca y qué no, en clase se dijo que se podía establecer un umbral $\epsilon$ tal que si
$d(x^{'},Hx)< \epsilon $ se acepta y sino se descarta. \newline

%----------------------------------------------------------------------------------------
%	Cuestión 13
%----------------------------------------------------------------------------------------
\section{Cual es el objetivo principal en el uso de la técnica RANSAC. Justificar
la respuesta}
El objetivo principal en el uso de RANSAC es determinar qué puntos, de entre todos los que ya tenemos, son
los revelantes/correctos para nuestro problema. Es decir, hacer la discriminación es función de lo que le indiquemos que
queremos estimar. \newline
Para lograrlo, lo que hace es muestrear y luego propone varias homografías y les pregunta al resto de puntos, que
votarán la que crean que es mejor. Finalmente, RANSAC devuelve la homografía que ha recibido más votaciones. \newline
Eso sí, hay que garantizar de que en toda la trayectoria hay, al menos, una H sin error. \newline
Por otra parte, tenemos que al discriminar RANSAC elimina puntos que no considera que son buenos pero que quizás si lo son.
Para intentar arreglar esto, se podría volver a lanzar todos los puntos descartados por la homografía, en lugar de por el matcher
usado antes(por ej, fuerza bruta) , pudiendo así recuperar algunos puntos que poder añadir a la lista de válidos.
%----------------------------------------------------------------------------------------
%	Cuestión 14
%----------------------------------------------------------------------------------------
\section{¿Si tengo 4 imágenes de una escena de manera que se solapan la 1-2, 2-3
y 3-4. ¿Cuál es el número mínimo de puntos en correspondencias necesarios para
montar un mosaico? Justificar la respuesta}
Creo que en ese caso, el número mínimo de puntos en correspondencias, aunque las imágenes estén solapadas, debe de ser
4 , por cada pareja de imágenes. Ésto es porque, como bien se sabe, 4 es el mínimo de puntos que deben darse \textbf{en correspondencias}
para poder calcular una homología válida. Por tanto, en este caso, si hay 4 imágenes necesitaríamos 12 puntos en correspondencias , 4 para
cada homología, suponiendo que la imágen 4 no engancha con la imágen 1.
%----------------------------------------------------------------------------------------
%	Cuestión 15
%----------------------------------------------------------------------------------------
\section{En la confección de un mosaico con proyección rectangular es esperable
que aparezcan deformaciones de la realidad. ¿Cuáles y porqué?.¿Bajo qué
condiciones esas deformaciones podrían desaparecer? Justificar la respuesta}
Teniendo en cuenta que en la confección de un mosaico se realiza la fusión geométrica y no fotométrica (al menos nosotros
la fotométrica no la hacemos), tendremos que una posible deformidad de la realidad será que las imágenes tendrán distintas tonalidades, pues
no suele ser que se tomen todas las fotos el mismo día y bajo las mismas condiciones de luminosidad. \newline
Otra posible deformación de la realidad que se me ocurre es que se pierde el paralelismo real que hay en la imágen, pues al adaptarlas
para que encajen entre sí estamos cambiando la proyectividad de estas, por lo que se pueden ver figuras más alargadas o rectangulares etc
que en las imágenes originales. \newline
Estas deformaciones podrían desaparecer,por ejemplo, en el caso de la luz, tomando las fotos bajo la misma luminosidad, o haciendo
una fusión fotométrica tras la geométrica, y en el caso de la proyectividad, no se puede eliminar, pero sí podríamos matizarla usando algoritmos más finos.  

\end{document}
